\documentclass[11pt,a4paper,sans]{moderncv}

% 1 - russian, 2 - english
\newcommand{\lang}[2]{#1} 

\moderncvtheme{classic}
\moderncvcolor{black}
\moderncvicons{awesome}

\usepackage[symbol]{footmisc} % footnotes.
\usepackage{cmap} % поиск в PDF
\usepackage[T2A]{fontenc} % кодировка
\usepackage[utf8]{inputenc} % кодировка исходного текста
\usepackage[english,russian]{babel} % локализация и переносы
\usepackage[unicode]{hyperref}
\usepackage[scale=0.8]{geometry}

\recomputelengths

\definecolor{linkcolour}{rgb}{0.02,0.27,0.68}
\hypersetup{colorlinks,breaklinks,urlcolor=linkcolour,linkcolor=linkcolour}

\setlength{\hintscolumnwidth}{3.5cm}

\firstname
  {\lang
    {Евгений}
    {Eugene}}
\lastname
  {\lang
    {Чельцов}
    {Cheltsov}}
\title
  {\lang
    {Разработчик}
    {Software Developer}}
\date{\today}
\email{chill.icp@gmail.com}
\address
  {\lang
    {Москва, Россия}
    {Moscow, Russia}}
\phone{+7 (***) ***-**-**}
\photo[64pt][0.4pt]{avatar}
\social[github]{ChillyBwoy}
\social[linkedin][www.linkedin.com/comm/in/chillybwoy]{Eugene Cheltsov}

% \extrainfo{Разработчик}

% \renewcommand{\rmdefault}{cmr} % Шрифт с засечками
\renewcommand{\sfdefault}{cmss} % Шрифт без засечек
\renewcommand{\ttdefault}{cmtt} % Моноширинный шрифт
\renewcommand{\labelitemi}{$\bullet$}

\begin{document}

\makecvtitle



\section
  {\lang
    {Опыт работы}
    {Work Experience}}
\cventry
  {2014--\lang
            {по н.в.}
            {Present}}
  {\lang
    {Архитектор программного обеспечения}
    {Software Architect}}
  {Mail.Ru Group}
  {\lang
    {Москва}
    {Moscow}}
  {}
  {Занимаюсь проектированием и разработкой инструментов, утилит и библиотек в медиапроектах Mail.Ru.}
\cvlistitem{Принимал участие в разработке проектов «\href{https://news.mail.ru}{Новости}», «\href{https://realty.mail.ru}{Недвижимость}», «\href{https://auto.mail.ru}{Авто}», «\href{https://hi-tech.mail.ru}{Hi-Tech}», «\href{https://lady.mail.ru}{Леди}», «\href{https://kino.mail.ru}{Кино}».\newline}

\cventry
  {2008--2014}
  {\lang
    {Программист}
    {Developer}}
  {Kitty Hug}
  {\lang
    {Москва}
    {Moscow}}
  {}
  {Разрабатывал фронтенд и бэкенд, занимался проектированием. Использовал Django, MySQL и React, писал тесты.}
\cvlistitem{Делал сайты для телеканалов СТС, ДТВ и Домашний, проекта «\href{http://itsmywine.ru}{It's My Wine}», интернет-магазина «Евродом», фотоаппаратов «Leica Camera», «Связного-клуба», сети поликлиник «Семейный Доктор», клуба «\href{http://propagandamoscow.com}{Пропаганда}», кафе «\href{http://filialmoscow.com/ru/}{Filial}» и многих других менее известных клиентов.}
\cvlistitem{Занимался интеграцией с платёжными системами(Assist, Payonline) и системами бронирования(Sabre, Sirena, Hotelbook, Академсервис).}
\cvlistitem{Разработал приложение \href{https://github.com/ChillyBwoy/django-plugshop}{django-plugshop}, на основе которого в дальнейшем было создано несколько интернет-магазинов.\newline}

\cventry
  {2011--2012}
  {\lang
    {JavaScript разработчик}
    {JavaScript Developer}}
  {Nextore}
  {\lang
    {Москва}
    {Moscow}}
  {}
  {}
\cvlistitem{Принимал участие в проектировании системы мониторинга радиоэфира «\href{http://dvhb.ru/potokfm}{Поток.FM}», занимался разработкой клиентской части сервиса. Использованные технологии: JavaScript(Prototype.js), Ruby(Sinatra), CSS.}
\cvlistitem{Разработал клиентскую часть интерактивной презентации для \href{http://nextore.ru/projects/3}{МЧС России}. Использованные технологии: Ruby(Sinatra), HAML, SCSS(Compass), CoffeeScript.}
\cvlistitem{В сжатые сроки реализовал прототип сервиса «\href{http://dvhb.ru/evomap}{Эвомап}» с использованием Ruby on Rails, CoffeeScript, Backbone и SCSS.\newline}

\cventry
  {2007--2008}
  {\lang
    {Программист}
    {Developer}}
  {Kenmore Design Ltd}
  {\lang
    {Бостон, Массачусетс, США}
    {Boston, MA, US}}
  {\lang
    {удалённая работа}
    {remote work}}
  {}
\cvlistitem{Разработал веб-фреймворк для создания сайтов на языке PHP, который был успешно использован при создании сайтов клиентов компании.}
\cvlistitem{Разработал CMS на основе Zend Framework.\newline}

\cventry
  {2005--2007}
  {\lang
    {3D художник}
    {3D Artist}}
  {4Reign Studios}
  {\lang
    {Курск}
    {Kursk}}
  {}
  {}
\cvlistitem{Принимал участие в создании концепции проекта.}
\cvlistitem{Занимался дизайном уровней, моделированием статичных объектов и персонажей для игр «\href{http://ru.akella.com/Game.aspx?id=354}{Dilemma}» и «\href{http://ru.akella.com/Game.aspx?id=1898}{Dilemma 2}», писал пользовательские утилиты для пакета 3D-моделирования Autodesk 3ds MAX на языке MAXScript.\newline}



\section
  {\lang
    {Технические навыки}
    {Technical Skills}}
\cvline
  {Главное}
  {TypeScript, JavaScript (Vanilla JS, React, Redux, jQuery, Backbone),\newline Python (Django, DRF, Graphene)}
\cvline
  {В свободное время}
  {Clojure/ClojureScript, интересуюсь Rust, Swift и мобильной разработкой}
\cvline
  {Когда-то давно}
  {Delphi, PHP, Ruby (Sinatra, RoR), C, C\texttt{++}, C\texttt{\#}}
\cvline
  {Верстаю}
  {CSS/SCSS/PostCSS, HTML/BEM}
\cvline
  {Использую}
  {Git, gulp, webpack, vim, vscode}
\cvline
  {Хочу получить опыт}
  {разработки коммерческих проектов на Rust}



\section
  {\lang
    {Образование}
    {Education}}
\subsection
  {\lang
    {Среднее образование}
    {Secondary school }}
\cventry
  {2001}
  {Гимназия № 8 им. академика Н.Н.Боголюбова}
  {г. Дубна Московской области}
  {}
  {}
  {}
\subsection
  {\lang
    {Высшее образование}
    {Higher education}}
\cventry
  {2002-2007}
  {Инженер по специальности «Программное обеспечение вычислительной техники и автоматизированных систем»}
  {Курский государственный технический университет (ныне ЮЗГУ)}
  {Курск}
  {}
  {}



\section
  {\lang
    {Языки}
    {Language Skills}}
\cvlanguage
  {\lang
    {Русский}
    {Russian}}
  {\lang
    {родной язык}
    {Native speaker}}
  {}
\cvlanguage
  {\lang
    {Английский}
    {English}}
  {\lang
    {читаю техническую литературу, могу поддержать разговор}
    {Good reading and translating ability}}
  {}
\cvlanguage
  {\lang
    {Испанский}
    {Spanish}}
  {\lang
    {уровень B1}
    {Nivel B1}}
  {}

\end{document}
