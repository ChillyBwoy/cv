\documentclass[11pt,a4paper,sans]{moderncv}

\moderncvtheme[blue]{classic}
\moderncvicons{awesome}

\usepackage[symbol]{footmisc} 						% footnotes.
\usepackage{cmap}								% поиск в PDF
\usepackage[T2A]{fontenc}				% кодировка
\usepackage[utf8]{inputenc}				% кодировка исходного текста
\usepackage[english,russian]{babel}	% локализация и переносы
\usepackage[unicode]{hyperref}
\usepackage[scale=0.8]{geometry}

\recomputelengths

\definecolor{linkcolour}{rgb}{0.17,0.36,0.63}
\hypersetup{colorlinks,breaklinks,urlcolor=linkcolour,linkcolor=linkcolour}

\setlength{\hintscolumnwidth}{3.5cm}


\firstname{Евгений}
\lastname{Чельцов}
\title{Разработчик}
\date{\today}
\email{eugene.cheltsov@icloud.com}
\address{}{Москва, Россия}
\mobile{+7 (962) 988-72-29}
\social[github]{ChillyBwoy}
%\homepage{brainstorage.me/ChillyBwoy}
%\extrainfo{Разработчик}

\renewcommand{\rmdefault}{cmr} 	% Шрифт с засечками
\renewcommand{\sfdefault}{cmss} 	% Шрифт без засечек
\renewcommand{\ttdefault}{cmtt} 	% Моноширинный шрифт
\renewcommand{\labelitemi}{$\bullet$}

\begin{document}

\makecvtitle

\section{Образование}
\subsection{Среднее образование}
\cvline{2001}{Гимназия № 8 им. академика Н.Н.Боголюбова г. Дубны Московской области}
\subsection{Высшее образование}
\cventry{2002-2007}{Инженер по специальности «Программное обеспечение вычислительной техники и автоматизированных систем»}{Курский государственный технический университет}
{}
{}
{}

\section{Опыт работы}
\cventry{Окт 2008--по н.в.}{Программист}{Kitty Hug}{Москва}{}{Разрабатываю фронт-энд и бэк-энд, занимаюсь проектированием. Использую Django, MySQL и React, пишу тесты.}

\cvlistitem{Делал сайты для телеканалов СТС, ДТВ и Домашний, интернет-магазина Евродом, фотоаппаратов «Leica Camera», Связного-клуба, сети поликлиник «Семейный Доктор», клуба «Пропаганда», кафе Filial и многих других менее известных клиентов.}

\cvlistitem{Интегрировался с API платёжных систем (Assist, Payonline) и систем бронирования (Sabre, Sirena, Hotelbook, Академсервис).}

\cvlistitem{Разработал приложение \href{https://github.com/ChillyBwoy/django-plugshop}{django-plugshop}, на основе которого в дальнейшем было создано несколько интернет-магазинов}

\cventry{Апр 2011--апр 2012}{JavaScript-разработчик}{Nextore}{Москва}{}{}
\cvlistitem{Принимал участие в проектировании системы мониторинга радиоэфира «\href{http://dvhb.ru/potokfm}{Поток.FM}», занимался разработкой клиентской части сервиса. Технологии: JavaScript(Prototype.js), Ruby(Sinatra), CSS}
\cvlistitem{Разработал клиентскую часть интерактивной презентации для \href{http://nextore.ru/projects/3}{МЧС России}. Технологии: HAML, SCSS(Compass), CoffeeScript}
\cvlistitem{В сжатые сроки реализовал прототип сервиса «\href{http://dvhb.ru/evomap}{Эвомап}» с использованием Ruby(Ruby on Rails), CoffeeScript, Haml и SCSS.}

\cventry{Сен 2007--окт 2008}{Программист}{Kenmore Design Ltd}{Бостон, Массачусетс, США}{удалённая работа}{}
\cvlistitem{Разработал веб-фреймворк для создания сайтов на языке PHP, который был успешно использован при создании сайтов клиентов компании.}
\cvlistitem{Разработал CMS на основе Zend Framework.}

\cventry{Май 2005--июн 2007}{3D-художник}{4Reign Studios}{Курск}{}{}
\cvlistitem{Принимал участие в создании концепции проекта.}
\cvlistitem{Занимался дизайном уровней, моделированием статичных объектов и персонажей для игр \href{http://ru.akella.com/Game.aspx?id=354}{Dilemma} и \href{http://ru.akella.com/Game.aspx?id=1898}{Dilemma 2}, писал пользовательские утилиты для пакета 3D-моделирования Autodesk 3ds MAX на языке MAXScript.}

\section{Навыки}
\cvline{Эксперт}{JavaScript/CoffeeScript, Python, CSS/SASS/SCSS, HTML/Haml}
\cvline{}{Ruby, Clojure, C, Objective-C, C++, bash}
\cvline{OS}{OS X, Linux(Debian, CentOS)}

\section{Иностранные языки}
\cvline{English}{Свободно читаю, могу поддержать разговор.}


\end{document}


