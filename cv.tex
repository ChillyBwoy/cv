\documentclass[11pt,a4paper,sans]{moderncv}

\moderncvtheme{classic}
\moderncvcolor{blue}
\moderncvicons{awesome}

\usepackage[symbol]{footmisc} % footnotes.
\usepackage[T1,T2A]{fontenc} % кодировка
\usepackage[utf8]{inputenc} % кодировка исходного текста
\usepackage[english,russian]{babel} % локализация и переносы
\usepackage[unicode]{hyperref}
\usepackage[scale=0.8]{geometry}

%\setlength{\hintscolumnwidth}{3cm}      % if you want to change the width of the column with the dates
%\setlength{\makecvtitlenamewidth}{10cm} % for the 'classic' style, if you want to force the width allocated to your name and avoid line breaks. be careful though, the length is normally calculated to avoid any overlap with your personal info; use this at your own typographical risks...

% 1 - russian, 2 - english
\newcommand{\lang}[2]{#2}

\setlength{\footskip}{66pt}

\definecolor{linkcolour}{rgb}{0.02,0.27,0.68}
\hypersetup{colorlinks,breaklinks,urlcolor=linkcolour,linkcolor=linkcolour}

\firstname
    {\lang
        {Евгений}
        {Eugene}}

\lastname
    {\lang
        {Чельцов}
        {Cheltcov}}

\title
    {\lang
        {Разработчик}
        {Senior Front-end Developer}}

\date{\today}
\email{***@***}
\address
    {\lang
        {Москва, Россия}
        {Moscow, Russia}}
\phone{+7 (***) ***-**-**}
% \photo[64pt][0.4pt]{avatar}
\social[github]{ChillyBwoy}
\social[linkedin][www.linkedin.com/comm/in/chillybwoy]{Eugene Cheltcov}

% \extrainfo{Разработчик}

% \renewcommand{\rmdefault}{cmr} % Шрифт с засечками
% \renewcommand{\sfdefault}{cmss} % Шрифт без засечек
% \renewcommand{\ttdefault}{cmtt} % Моноширинный шрифт
\renewcommand{\listitemsymbol}{-~}

\begin{document}

\makecvtitle

\section
    {\lang
        {Опыт работы}
        {Experience}}

\cventry
    {\lang{Июл}{Jul} 2018\\\lang{по н.в.}{Present}}
    {\lang
        {Фронтенд разработчик}
        {Senior Front-end developer}}
    {SuperJob.ru}
    {\lang
        {Москва}
        {Moscow}}
    {}
    {\lang
        {Разработка и поддержка сайта компании, создание новых инструментов, утилит и библиотек.}
        {Design and development of company superjob.ru, utilities and libraries}}


\cventry
    {\lang{Дек}{Dec} 2014\\\lang{Июл}{Jul} 2018}
    {\lang
        {Архитектор программного обеспечения}
        {Software Architect}}
    {Mail.Ru Group}
    {\lang
        {Москва}
        {Moscow}}
    {}
    {\lang
        {Занимаюсь проектированием и разработкой инструментов, утилит и библиотек в медиапроектах Mail.Ru.}
        {Designed and developed various tools, utilities and libraries for Media Projects Mail.Ru}}


\cvlistitem
    {\lang
        {Принимал участие в разработке проектов \href{https://news.mail.ru}{news.mail.ru}, \href{https://realty.mail.ru}{realty.mail.ru}, \href{https://auto.mail.ru}{auto.mail.ru}, \href{https://hi-tech.mail.ru}{hi-tech.mail.ru}, \href{https://lady.mail.ru}{lady.mail.ru}, \href{https://kino.mail.ru}{kino.mail.ru}}
        {Actively participated in development of such projects as \href{https://news.mail.ru}{news.mail.ru}, \href{https://realty.mail.ru}{realty.mail.ru}, \href{https://auto.mail.ru}{auto.mail.ru}, \href{https://hi-tech.mail.ru}{hi-tech.mail.ru}, \href{https://lady.mail.ru}{lady.mail.ru}, \href{https://kino.mail.ru}{kino.mail.ru}}}

\cvlistitem
    {\lang
        {Разработал фреймворк для создания клиентских приложений с использованием языка TypeScript}
        {Developed proprietary client-side framework using TypeScript}}

\cvlistitem
    {\lang
        {TODO}
        {Designed and developed a set of utilities to manage push-notifications for Chrome browser}}

        

\cventry
    {\lang{Окт}{Oct} 2008\\\lang{Дек}{Dec} 2014}
    {\lang
        {Программист}
        {Software Developer}}
    {Kitty Hug}
    {\lang
        {Москва}
        {Moscow}}
    {}
    {\lang
        {Занимался разработкой и проектированием как серверной, так и клиентской части проектов компании. Использовал Django, Django-CMS, Tornado, MySQL, JQuery и React, занимался написанием юнит-тестов.}
        {Designed and developed both server and client side of company projects. Used such tools as Django web framework, Django-CMS, Tornado, MySQL, JQuery and React}}

\cvlistitem
    {\lang
        {Делал сайты для телеканалов СТС, ДТВ и Домашний, проекта «It's My Wine»(\href{http://itsmywine.ru}{itsmywine.ru}), интернет-магазина «Евродом», фотоаппаратов «Leica Camera», «Связного-клуба», сети поликлиник «Семейный Доктор», клуба «Пропаганда», кафе «Filial»(\href{http://filialmoscow.com/ru/}{filialmoscow.com}), торгово-офисного центра «Гименей»(\href{http://himeney.ru}{himeney.ru}) и многих других менее известных клиентов.}
        {Made websites}}

\cvlistitem
    {\lang
        {Занимался интеграцией с платёжными системами Assist и Payonline, системами бронирования авиабилетов Sabre и Sirena, системами бронирования отелей Hotelbook и Академсервис.}
        {TODO}}

\cvlistitem
    {\lang
        {Разработал приложение \href{https://github.com/ChillyBwoy/django-plugshop}{django-plugshop}, на основе которого в дальнейшем было создано несколько интернет-магазинов, например deathstar.ru и \href{http://powerball.ru}{powerball.ru}}
        {TODO}}


\cventry
    {\lang{Июл}{Jul} 2010\\\lang{Дек}{Dec} 2012}
    {\lang
        {Фронтенд разработчик}
        {Front-end developer}}
    {Nextore}
    {\lang
        {Москва}
        {Moscow}}
    {}
    {\lang
        {Разработка проектов компании}
        {Development of different projects}}

\cvlistitem
    {\lang
        {Принимал участие в проектировании системы мониторинга радиоэфира «Поток.FM»(\href{https://dvhb.ru/potokfm}{dvhb.ru/potokfm}), занимался разработкой клиентской части сервиса. Использованные технологии: JavaScript(Prototype.js), Ruby(Sinatra), CSS.}
        {Participated in design of system for monitoring of radio broadcasting, designed front-end part of the project. «Potok.FM»(\href{https://dvhb.ru/en/potokfm}{dvhb.ru/potokfm})}}

\cvlistitem
    {\lang
        {Разработал клиентскую часть интерактивной презентации для \href{http://nextore.ru/projects/3}{МЧС России}. Использованные технологии: Ruby(Sinatra), HAML, SCSS(Compass), CoffeeScript}
        {Developed client-side of \href{http://nextore.ru/projects/3}{interactive presentation} for Ministry of Emergency Situations using "Sinatra(Ruby)" framework and CoffeeScript language}}

\cvlistitem
    {\lang
        {В сжатые сроки реализовал прототип сервиса «\href{https://dvhb.ru/evomap}{Эвомап}» с использованием Ruby on Rails, CoffeeScript, Backbone и SCSS}
        {In deadlines had developed an interactive prototype of geolocation service "\href{https://dvhb.ru/evomap}{Evomap}" using "Ruby on Rails" framework, CoffeeScript language and Backbone framework}}


\cventry
    {\lang{Окт}{Oct} 2007\\\lang{Окт}{Oct} 2008}
    {\lang
        {Программист}
        {Developer}}
    {Kenmore Design Ltd}
    {\lang
        {Бостон, Массачусетс, США}
        {Boston, MA, US}}
    {\lang
        {удалённая работа}
        {remote work}}
    {}

\cvlistitem
    {\lang
        {Разработал веб-фреймворк для создания сайтов на языке PHP, который был успешно использован при создании проектов клиентов компании.}
        {Developed a proprietary web-framework using PHP language, which had been successfully integrated with clients projects}}

\cvlistitem
    {\lang
        {Разработал CMS на основе Zend Framework(PHP).}
        {Designed and developed CMS on top of Zend Framework(PHP)}}

\cventry
    {\lang{Май}{May} 2005\\\lang{Авг}{Aug} 2007}
    {\lang
        {3D художник}
        {3D Artist}}
    {4Reign Studios}
    {\lang
        {Курск}
        {Kursk}}
    {}
    {}

\cvlistitem
    {\lang
        {Принимал участие в создании концепции проекта}
        {Participated in creation of game concept}}

\cvlistitem
    {\lang
        {Занимался дизайном уровней, моделированием статичных объектов и персонажей для игр «\href{https://www.igromania.ru/game/3494/Dilemma.html}{Dilemma}» и «\href{https://www.igromania.ru/game/3959/Dilemma_2.html}{Dilemma 2}», писал пользовательские утилиты для пакета 3D-моделирования Autodesk 3ds MAX на языке MAXScript.}
        {Designed game levels, static environment objects and characters for such games as "\href{https://www.igromania.ru/game/3494/Dilemma.html}{Dilemma}" and "\href{https://www.igromania.ru/game/3959/Dilemma_2.html}{Dilemma 2}". Developed automatisation scripts for "Autodesk 3ds Max" using MAXScript language }}

\section
    {\lang
        {Технические навыки}
        {Technical Skills}}

\cvline
    {\lang
        {Главное}
        {Main}}
    {TypeScript, JavaScript (Vanilla JS, React, jQuery),\newline Python (Django, DRF, Graphene)}

\cvline
    {\lang
        {Использую инструменты и технологии}
        {Tools and Technologies}}
    {CSS, PostCSS, SCSS,\newline HTML, BEM,\newline Git, gulp, webpack, bash}

\cvline
    {\lang
        {Имею небольшой опыт}
        {Have some experience}}
    {Clojure, ClojureScript, Swift}

\cvline
    {\lang
        {В свободное время}
        {In spare time}}
    {\lang
        {интересуюсь Rust}
        {Have an interest in Rust programming language and mobile apps(iOS) development}}

\cvline
    {\lang
        {В прошлом}
        {In the past}}
    {PHP, Ruby(RoR, Sinatra)}

\section
    {\lang
        {Образование}
        {Education}}

\subsection
    {\lang
        {Высшее образование}
        {University}}

\cventry
    {2002-2007}
    {\lang
        {Инженер по специальности «Программное обеспечение вычислительной техники и автоматизированных систем»}
        {Software of computer facilities and automated systems}}
    {\lang
        {«Юго-Западный государственный университет» (бывший «Курский государственный технический университет»)}
        {South-West State University (ex. Kursk State Technical University)}}
    {\lang
        {Курск}
        {Kursk}}
    {}
    {}

\section
    {\lang
        {Языки}
        {Language Skills}}

\cvlanguage
    {\lang
        {Русский}
        {Russian}}
    {\lang
        {родной язык}
        {Native speaker}}
    {}

\cvlanguage
    {\lang
        {Английский}
        {English}}
    {\lang
        {читаю техническую литературу, могу поддержать разговор}
        {Good reading and translating ability}}
    {}

\cvlanguage
    {\lang
        {Испанский}
        {Spanish}}
    {\lang
        {уровень B1}
        {Nivel B1}}
    {}

\end{document}
