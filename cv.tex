\documentclass[11pt,a4paper,sans]{moderncv}

\moderncvtheme[black]{classic}
\moderncvicons{awesome}

\usepackage[symbol]{footmisc} 						% footnotes.
\usepackage{cmap}								% поиск в PDF
\usepackage[T2A]{fontenc}				% кодировка
\usepackage[utf8]{inputenc}				% кодировка исходного текста
\usepackage[english,russian]{babel}	% локализация и переносы
\usepackage[unicode]{hyperref}
\usepackage[scale=0.8]{geometry}

\recomputelengths

\definecolor{linkcolour}{rgb}{0.17,0.36,0.63}
\hypersetup{colorlinks,breaklinks,urlcolor=linkcolour,linkcolor=linkcolour}

\setlength{\hintscolumnwidth}{3.5cm}


\firstname{Евгений}
\lastname{Чельцов}
\title{Разработчик}
\date{\today}
\email{masta\_chill@mail.ru}
\address{}{Москва, Россия}
\mobile{+7 (962) 988-72-29}
\social[github]{ChillyBwoy}

%\homepage{brainstorage.me/ChillyBwoy}
%\extrainfo{Разработчик}

\renewcommand{\rmdefault}{cmr} 	% Шрифт с засечками
\renewcommand{\sfdefault}{cmss} 	% Шрифт без засечек
\renewcommand{\ttdefault}{cmtt} 	% Моноширинный шрифт
\renewcommand{\labelitemi}{$\bullet$}

\begin{document}

\makecvtitle

\section{Опыт работы}
\cventry{Дек 2014--по н.в.}{Программист JavaScript}{Mail.Ru Group}{Москва}{}
{Разрабатываю фронтенд.}
\cvlistitem{Разработка компонентов внутреннего фреймфорка.\newline}
\cvlistitem{Принимал участие в разработке проектов «\href{https://news.mail.ru}{Новости}», «\href{https://realty.mail.ru}{Недвижимость}», «\href{https://auto.mail.ru}{Авто}», «\href{https://hi-tech.mail.ru}{Hi-Tech}», «\href{https://lady.mail.ru}{Леди}», «\href{https://afisha.mail.ru}{Афиша}».\newline}

\cventry{Окт 2008--дек 2014}{Программист}{Kitty Hug}{Москва}{}{Разрабатывал фронтенд и бэкенд, занимался проектированием. Использовал Django, MySQL и React, писал тесты(иногда).}

\cvlistitem{Делал сайты для телеканалов СТС, ДТВ и Домашний, проекта «\href{http://itsmywine.ru}{It's My Wine}», интернет-магазина «Евродом», фотоаппаратов «Leica Camera», «Связного-клуба», сети поликлиник «Семейный Доктор», клуба «\href{http://propagandamoscow.com}{Пропаганда}», кафе «\href{http://filialmoscow.com/ru/}{Filial}» и многих других менее известных клиентов.}

\cvlistitem{Занимался интеграцией с платёжными системами(Assist, Payonline) и системами бронирования(Sabre, Sirena, Hotelbook, Академсервис).}

\cvlistitem{Разработал приложение \href{https://github.com/ChillyBwoy/django-plugshop}{django-plugshop}, на основе которого в дальнейшем было создано несколько интернет-магазинов.\newline}

\cventry{Апр 2011--апр 2012}{JavaScript-разработчик}{Nextore}{Москва}{}{}
\cvlistitem{Принимал участие в проектировании системы мониторинга радиоэфира «\href{http://dvhb.ru/potokfm}{Поток.FM}», занимался разработкой клиентской части сервиса. Использованные технологии: JavaScript(Prototype.js), Ruby(Sinatra), CSS.}
\cvlistitem{Разработал клиентскую часть интерактивной презентации для \href{http://nextore.ru/projects/3}{МЧС России}. Использованные технологии: Ruby(Sinatra), HAML, SCSS(Compass), CoffeeScript.}
\cvlistitem{В сжатые сроки реализовал прототип сервиса «\href{http://dvhb.ru/evomap}{Эвомап}» с использованием Ruby on Rails, CoffeeScript, Backbone и SCSS.\newline}

\cventry{Сен 2007--окт 2008}{Программист}{Kenmore Design Ltd}{Бостон, Массачусетс, США}{удалённая работа}{}
\cvlistitem{Разработал веб-фреймворк для создания сайтов на языке PHP, который был успешно использован при создании сайтов клиентов компании.}
\cvlistitem{Разработал CMS на основе Zend Framework.\newline}

\cventry{Май 2005--июн 2007}{3D-художник}{4Reign Studios}{Курск}{}{}
\cvlistitem{Принимал участие в создании концепции проекта.}
\cvlistitem{Занимался дизайном уровней, моделированием статичных объектов и персонажей для игр «\href{http://ru.akella.com/Game.aspx?id=354}{Dilemma}» и «\href{http://ru.akella.com/Game.aspx?id=1898}{Dilemma 2}», писал пользовательские утилиты для пакета 3D-моделирования Autodesk 3ds MAX на языке MAXScript.\newline}

\section{Навыки}
\cvline{В основном}{JavaScript/CoffeeScript (Vanilla JS, React, jQuery, Backbone+Marionette), \newline Python (Django, Django REST Framework)}
\cvline{Очень редко}{Ruby (Sinatra, Ruby on Rails), bash}
\cvline{На досуге}{лениво изучаю Clojure/ClojureScript}
\cvline{Когда-то давно}{Pascal, C, C\texttt{++}, Delphi, C\texttt{\#}, PHP}
\cvline{Верстаю}{CSS/SCSS, HTML/Haml/fest}
\cvline{Использую}{Git, vim, Sentry, gulp, webpack}

\section{Образование}
\subsection{Среднее образование}
\cvline{2001}{Гимназия № 8 им. академика Н.Н.Боголюбова г. Дубны Московской области\newline}
\subsection{Высшее образование}
\cventry{2002-2007}{Инженер по специальности «Программное обеспечение вычислительной техники и автоматизированных систем»}{Курский государственный технический университет}
{}
{}
{}

\section{Иностранные языки}
\cvline{English}{Свободно читаю, могу поддержать разговор.}
\cvline{Español}{Начальный.}


\end{document}


