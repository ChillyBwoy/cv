\documentclass[11pt,a4paper,sans]{moderncv}

% 1 - russian, 2 - english
\newcommand{\lang}[2]{#1}

\moderncvtheme{classic}
\moderncvcolor{black}
\moderncvicons{awesome}

\usepackage[symbol]{footmisc} % footnotes.
\usepackage{cmap} % поиск в PDF
\usepackage[T1,T2A]{fontenc} % кодировка
\usepackage[utf8]{inputenc} % кодировка исходного текста
\usepackage[english,russian]{babel} % локализация и переносы
\usepackage[unicode]{hyperref}
\usepackage[scale=0.8]{geometry}

\recomputelengths

\definecolor{linkcolour}{rgb}{0.02,0.27,0.68}
\hypersetup{colorlinks,breaklinks,urlcolor=linkcolour,linkcolor=linkcolour}

\setlength{\hintscolumnwidth}{3.5cm}

\firstname
    {\lang
        {Евгений}
        {Eugene}}

\lastname
    {\lang
        {Чельцов}
        {Cheltsov}}
\title
    {\lang
        {Разработчик}
        {Software Developer}}
\date{\today}
\email{chill.icp@gmail.com}
\address
    {\lang
        {Москва, Россия}
        {Moscow, Russia}}
\phone{+7 (***) ***-**-**}
% \photo[64pt][0.4pt]{avatar}
\social[github]{ChillyBwoy}
\social[linkedin][www.linkedin.com/comm/in/chillybwoy]{Eugene Cheltsov}

% \extrainfo{Разработчик}

% \renewcommand{\rmdefault}{cmr} % Шрифт с засечками
\renewcommand{\sfdefault}{cmss} % Шрифт без засечек
\renewcommand{\ttdefault}{cmtt} % Моноширинный шрифт
\renewcommand{\labelitemi}{$\bullet$}

\begin{document}

\makecvtitle


\section
    {\lang
        {Опыт работы}
        {Work Experience}}


\cventry
    {2014--\lang
        {по н.в.}
        {Present}}
    {\lang
        {Архитектор программного обеспечения}
        {Software Architect}}
    {Mail.Ru Group}
    {\lang
        {Москва}
        {Moscow}}
    {}
    {\lang
        {Занимаюсь проектированием и разработкой инструментов, утилит и библиотек в медиапроектах Mail.Ru.}
        {Design and develop tools, utilities and libraries for Media Projects Mail.Ru}}

\cvlistitem
    {\lang 
        {Принимал участие в разработке проектов \href{https://news.mail.ru}{news.mail.ru}, \href{https://realty.mail.ru}{realty.mail.ru}, \href{https://auto.mail.ru}{auto.mail.ru}, \href{https://hi-tech.mail.ru}{hi-tech.mail.ru}, \href{https://lady.mail.ru}{lady.mail.ru}, \href{https://kino.mail.ru}{kino.mail.ru}}
        {Actively participated in development of such projects as \href{https://news.mail.ru}{news.mail.ru}, \href{https://realty.mail.ru}{realty.mail.ru}, \href{https://auto.mail.ru}{auto.mail.ru}, \href{https://hi-tech.mail.ru}{hi-tech.mail.ru}, \href{https://lady.mail.ru}{lady.mail.ru}, \href{https://kino.mail.ru}{kino.mail.ru}}\newline}


\cventry
    {2008--2014}
    {\lang
        {Программист}
        {Software Developer}}
    {Kitty Hug}
    {\lang
        {Москва}
        {Moscow}}
    {}
    {\lang
        {Занимался разработкой и проектированием как серверной, так и клиентской части проектов компании. Использовал Django, Django-CMS, Tornado, MySQL, JQuery и React, занимался написанием юнит-тестов.}
        {Designed and developed both server and client side of company projects. Used such tools as Django web framework, Django-CMS, Tornado, MySQL, JQuery and React}}

\cvlistitem
    {\lang
        {Делал сайты для телеканалов СТС, ДТВ и Домашний, проекта «It's My Wine»(\href{http://itsmywine.ru}{itsmywine.ru}), интернет-магазина «Евродом», фотоаппаратов «Leica Camera», «Связного-клуба», сети поликлиник «Семейный Доктор», клуба «Пропаганда», кафе «Filial»(\href{http://filialmoscow.com/ru/}{filialmoscow.com}), торгово-офисного центра «Гименей»(\href{http://himeney.ru}{himeney.ru}) и многих других менее известных клиентов.}
        {Made websites}}

\cvlistitem
    {\lang
        {Занимался интеграцией с платёжными системами Assist и Payonline, системами бронирования авиабилетов Sabre и Sirena, системами бронирования отелей Hotelbook и Академсервис.}
        {TODO}}

\cvlistitem
    {\lang
        {Разработал приложение \href{https://github.com/ChillyBwoy/django-plugshop}{django-plugshop}, на основе которого в дальнейшем было создано несколько интернет-магазинов, например deathstar.ru и \href{http://powerball.ru}{powerball.ru}}
        {Design applications ... TODO}\newline}


\cventry
    {2011--2012}
    {\lang
        {Фронтенд разработчик}
        {Front End Developer}}
    {Nextore}
    {\lang
        {Москва}
        {Moscow}}
    {}
    {\lang
        {Разработка}
        {Development}}

\cvlistitem
    {\lang
        {Принимал участие в проектировании системы мониторинга радиоэфира «Поток.FM»(\href{https://dvhb.ru/potokfm}{dvhb.ru/potokfm}), занимался разработкой клиентской части сервиса. Использованные технологии: JavaScript(Prototype.js), Ruby(Sinatra), CSS.}
        {Participated in design of system for monitoring of radio broadcasting, designed front-end part of the project. «Potok.FM»(\href{https://dvhb.ru/en/potokfm}{dvhb.ru/potokfm})}}

\cvlistitem
    {\lang
        {Разработал клиентскую часть интерактивной презентации для \href{http://nextore.ru/projects/3}{МЧС России}. Использованные технологии: Ruby(Sinatra), HAML, SCSS(Compass), CoffeeScript.}
        {TODO}}

\cvlistitem
    {\lang
        {В сжатые сроки реализовал прототип сервиса «\href{http://dvhb.ru/evomap}{Эвомап}» с использованием Ruby on Rails, CoffeeScript, Backbone и SCSS.}
        {TODO}\newline}


\cventry
    {2007--2008}
    {\lang
        {Программист}
        {Developer}}
    {Kenmore Design Ltd}
    {\lang
        {Бостон, Массачусетс, США}
        {Boston, MA, US}}
    {\lang
        {удалённая работа}
        {remote work}}
    {}

\cvlistitem
    {\lang
        {Разработал веб-фреймворк для создания сайтов на языке PHP, который был успешно использован при создании проектов клиентов компании.}
        {TODO}}

\cvlistitem
    {\lang
        {Разработал CMS на основе Zend Framework.}
        {TODO}\newline}


\cventry
    {2005--2007}
    {\lang
        {3D художник}
        {3D Artist}}
    {4Reign Studios}
    {\lang
        {Курск}
        {Kursk}}
    {}
    {}

\cvlistitem
    {\lang
        {Принимал участие в создании концепции проекта.}
        {TODO}}

\cvlistitem
    {\lang
        {Занимался дизайном уровней, моделированием статичных объектов и персонажей для игр «\href{https://www.igromania.ru/game/3494/Dilemma.html}{Dilemma}» и «\href{https://www.igromania.ru/game/3959/Dilemma_2.html}{Dilemma 2}», писал пользовательские утилиты для пакета 3D-моделирования Autodesk 3ds MAX на языке MAXScript.}
        {TODO}\newline}


\section
    {\lang
        {Технические навыки}
        {Technical Skills}}

\cvline
    {\lang
        {Главное}
        {Main}}
    {TypeScript, JavaScript (Vanilla JS, React, Redux, jQuery, Backbone),\newline Python (Django, DRF, Graphene)}

\cvline
{\lang
    {Имею небольшой опыт}
    {Have some experience}}
{Clojure, ClojureScript, Swift}

\cvline
    {\lang
        {В свободное время}
        {TODO}}
    {\lang
        {интересуюсь Rust и Haskell}
        {Have an interest in Rust and Haskell}}

\cvline
    {\lang
        {Когда-то давно}
        {TODO}}
    {Delphi, PHP, Java, Ruby (RoR, Sinatra), C, C\texttt{++}, C\texttt{\#}}

\cvline
    {\lang
        {Верстаю}
        {TODO}}
    {CSS, PostCSS, SCSS, HTML/BEM}

\cvline
    {\lang
        {Использую}
        {TODO}}
    {Git, gulp, webpack, vim, vscode}


\section
    {\lang
        {Образование}
        {Education}}


\subsection
    {\lang
        {Среднее образование}
        {Secondary school }}

\cventry
    {2001}
    {\lang
        {Гимназия № 8 им. академика Н.Н.Боголюбова}
        {TODO}}
    {\lang
        {г. Дубна Московской области}
        {Dubna}}
    {}
    {}
    {}


\subsection
    {\lang
        {Высшее образование}
        {Higher Education}}

\cventry
    {2002-2007}
    {\lang
        {Инженер по специальности «Программное обеспечение вычислительной техники и автоматизированных систем»}
        {Software of computer facilities and automated systems}}
    {\lang
        {«Юго-Западный государственный университет» (бывший «Курский государственный технический университет»)}
        {South-West State University (ex. Kursk State Technical University)}}
    {\lang
        {Курск}
        {Kursk}}
    {}
    {}

\section
    {\lang
        {Языки}
        {Language Skills}}

\cvlanguage
    {\lang
        {Русский}
        {Russian}}
    {\lang
        {родной язык}
        {Native speaker}}
    {}

\cvlanguage
    {\lang
        {Английский}
        {English}}
    {\lang
        {читаю техническую литературу, могу поддержать разговор}
        {Good reading and translating ability}}
    {}

\cvlanguage
    {\lang
        {Испанский}
        {Spanish}}
    {\lang
        {уровень B1}
        {Nivel B1}}
    {}

\end{document}
